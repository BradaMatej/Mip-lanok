% Metódy inžinierskej práce

\documentclass[10pt,slovak,a4paper]{article}
\usepackage[slovak]{babel}
%\usepackage[T1]{fontenc}
\usepackage[IL2]{fontenc} % lepšia sadzba písmena Ľ než v T1
\usepackage[utf8]{inputenc}
\usepackage{graphicx}
\usepackage{url} % príkaz \url na formátovanie URL
\usepackage{hyperref} % odkazy v texte budú aktívne (pri niektorých triedach dokumentov spôsobuje posun textu)

\usepackage{cite}
%\usepackage{times}



\title{Úloha počítačových hier a gamifikácie v študentskom učení
\thanks{Semestrálny projekt v predmete Metódy inžinierskej práce, ak. rok 2022/23, vedenie: MSc. Mirwais Ahmadzai}} % meno a priezvisko vyučujúceho na cvičeniach

\author{Matej Brada\\[2pt]
	{\small Slovenská technická univerzita v Bratislave}\\
	{\small Fakulta informatiky a informačných technológií}\\
	{\small \texttt{xbrada@stuba.sk}}
	}

\date{\small 6. november 2022}

\begin{document}

\maketitle

\begin{abstract}
BRADA, M.: Úloha počítačových hier a gamifikácie v študentskom učení
\\\\
U každého študenta je hra veľmi obľúbená. Počítačové hry sú súčasťou života každého mladého človeka a v dôsledku toho, sa hranie hier v učení študentov ukázalo ako účinný spôsob inšpirácie a motivácie študentov vzdelávať sa. (Salen, K.) Špeciálne gamifikácia, teda
technika, ktorá využíva herné prvky, môže učenie študentov rôznych vekových kategóriách
obohatiť. Práve preto som si vybral tému mojej práce Úloha počítačových hier a gamifikácie v študentskom učení. Nasmerovali ma k tomu články od autorov Katie Salen, Richard Blankman, Matej Held ktorí mi priblížili učenie založené na hrách. V práci sa zameriam na úlohu počítačových hrier a gamifikácie v učení študentov a na využívanie nástrojov gamifikácie – prvkov hier.
\\\\
	Kľúčové slová: Počitačová hra, gamifikácia, nástroje gamifikácie, vyžitie gamifikácie, vzdelávanie, škola. 

\end{abstract}
\newpage
\renewcommand*\contentsname{Obsah}
\maketitle
\tableofcontents
\newpage
\section{Úvod}
\noindent
\hspace*{1.5em}
Študenti počítačové hry obľubujú a sú súčasťou ich života. Práve preto som chcel prísť na to, či online hry, ktoré sú dnes neoddeliteľnou súčasťou študenta, vedia pomôcť v jeho učení sa. Zistil som, že v rámci učenia sa prostredníctvom počítačových hier, je gamifikácia ako technika účinným nástrojom. Autori ako Katie Salen, Richard Blantman, či Matej Held sa danou problematikou zaoberajú a priblížili mi učenie založené na hrách. \\

\section{Počítačové hry a gamifikácia z pohľadu rôznych autorov} \label{ina}
\noindent
\hspace*{1.5em}
Skôr ako začnem riešiť problematiku úlohy počítačových hier a gamifikácie v učení študentov a využívanie nástrojov gamifikácie – prvkov hier v učení študentov, je potrebné zadefinovať pojmy počítačová hra a gamifikácia. Danej problematike sa venovali viacerí autori. Napríklad Jirí Dostál v článku Výukový software a didaktické počítačové hry – nástroje moderního definuje didaktickú počítačovú hru ako „softvér umožňujúci zábavnou formou navodiť činnosti zamerané na rozvoj osobnosti jedinca.“ \cite{cite10}

%\footnote[1]{\footnotesize DOSTÁL, J.: INSTRUCTIONAL SOFTWARE AND COMPUTER GAMES – TOOLS OF MODERN EDUCATION. In Journal of Technology and Information Education, 2009, roč. 1, č. 1, s. 27. [online]. [2022-010-26]. Dostupné na: <https://pdfs.semanticscholar.org/3d10/dc1263710474cf692e271807550ecac82e4b.pdf$\_$ga2.32792574.1730075480.1666703936-1705427111.1666703936=>.} 


Teda počítačová hra, alebo inak povedané videohra majú, či už v dospievaní mladých ľudí, ale aj v živote človeka dôležitú úlohu. Hra je jedna veľká „záhada“, resp. problém, ktorý vyriešime prostredníctvom čiastkových úloh. Dané úlohy sú v ideálnom prípade stále viac a viac náročnejšie tak, aby hráča po celý čas zabávala a nefrustrovala. V hre je dôležitý problém, ktorý by mal obsahovať viacero riešení, simulovať, teda napodobňovať komplikované životné situácie, ktorým bude dieťa čeliť v dospelom živote.
\cite{cite2}
%\footnote[2]{  BOŽÍK, M.: Dovolil som deťom v škole hrať videohry. Pozrite, čo sa stalo. [online]. [2022-10-27]. Dostupné na: <https://dennikn.sk/blog/1692272/videohry-vo-vzdelavani/>.}\\


\indent Podľa výskumu (Božík, 2012) až 60$\%$ hráčov herného žánru Multiplayer Online Role-Playing Game uviedlo, že práve hranie hier im zabezpečilo vyššiu úroveň cudzieho jazyka, nakoľko kyberpriestor je vo veľkej miere v anglickom jazyku a práve to je pridaná hodnota videohrier. \cite{cite2}

%\footnote[3]{BOŽÍK, M.: Videohry v kontexte vyučovacieho procesu. [online]. [2022-10-27]. Dostupné na: < https://www.direktor.sk/sk/casopis/manazment-skoly-v-praxi/videohry-v-kontexte-vyucovacieho-procesu.m-237.html>.} 

Je to spôsobené tým, že väčšina známych videohier je v anglickom jazyku. Ak je hra vystavaná tak, ako som uvádzal vyššie, čo znamená, že v nej riešime veľký problém, pozostávajúci z malých úloh, ktoré motivujú študenta k ich vyriešeniu, tak je nútení si anglické výrazy vyhľadávať a naučiť sa ich.\\
\indent Aká je definícia gamifikácie? Aký je rozdiel medzi počítačovou hrou a gamifikáciou? Podľa Mateja Heldu „ide o využívanie herných mechaník v neherných oblastiach s cieľom odľahčiť určitú aktivitu a motivovať ľudí, aby sa do nej s nadšením pustili.“
\cite{cite4}
%\footnote[4]{HELD, M.: Hravosť vo vzdelávaní – gamifikácia. [online]. [2022-10-28]. Dostupné na: <https://vlcata.sk/hravost-vo-vzdelavani-gamifikacia/>.}  

V praxi to znamená, že aplikujeme len herný dizajn a niektoré prvky hier do vzdelávacieho prostredia, čím je pre študentov výučba pútavejšia.
\cite{cite5}
%\footnote[5]{  BLANKMAN, R.: The Fun of Learning: Gamification in Education. [online]. [2022-10-28]. Dostupné na: <https://www.hmhco.com/blog/what-is-gamification-in-education>.}\\

\indent Medzi takéto prvky môže patriť bodovanie, rôzne odznaky a podobne. Práve to môže byť pre študentov zaujímavejšie. Gamifikácia je výborným prostriedkom na budovanie motivácie, pomocou ktorej u študentov zvyšujeme záujem vo vyučovaní. V procese učenia sa, sa tým redukujú problémy spôsobené študentmi vo vyučovacom procese. \cite{cite5}

%\footnote[6]{BLANKMAN, R.: The Fun of Learning: Gamification in Education. [online]. [2022-10-28]. Dostupné na: <https://www.hmhco.com/blog/what-is-gamification-in-education>.}\\


\indent V reálnom živote a v rôznych aplikáciách sa tiež môžeme stretnúť s gamifikáciou. 
\begin{itemize}
\item 	Waze – navigačná aplikácia 
\item	Duolingo – učenie 
\item	Runkeeper – športová 
\item LinkedIn – sociálna sieť 
\cite{cite6}
%\footnote[7]{URBANOVÁ, E.: Čo je to gamifikácia a oplatí sa ju použiť? . [online]. [2022-10-29]. Dostupné na: <https://www.e-learnmedia.sk/blog/co-je-to-gamifikacia-a-oplati-sa-ju-pouzit/ >.}


\end{itemize} 
\noindent
\hspace*{1.5em}
 Ako uvádzam, problematike počítačových hier a gamifikácii sa venujú viacerí autori. Výskumy ukazujú, že či už počítačová hra, alebo gamifikácia môže v prípade správneho použitia výrazne obohatiť študentov tak v reálnom živote ako aj v študentskom učení. Práve preto si myslím, že úloha či už počítačových hier, alebo gamifikácie je dôležitá pre pri učení študentov.




\section{Počítačové hry a ich úloha v študentskom učení } 
\label{dolezita} 
\noindent
\hspace*{1.5em}
Podľa J. Dostála „hra prostredníctvom počítača je činnosť jedinca, ktorá má podstatu vo virtuálnom prostredí simulovanom počítačom a primárne spočíva v rozvoji osobnosti, pričom podľa svojho zamerania môže poskytovať zábavu, odreagovanie a relaxáciu.
\cite{cite12}
%\footnote[8]{DOSTÁL, J.: INSTRUCTIONAL SOFTWARE AND COMPUTER GAMES – TOOLS OF MODERN EDUCATION. In Journal of Technology and Information Education, 2009, roč. 1, č. 1, s. 27. [online]. [2022-010-26]. Dostupné na: <https://pdfs.semanticscholar.org/3d10/dc1263710474cf692e271807550ecac82e4b.pdf?$\_$ga=2.32792574.1730075480.1666703936-1705427111.1666703936>.} \\


 \indent Ak sa študent bude pravidelne pohybovať v prostredí počítačovej hry, kde aktívne využíva napr. anglický jazyk úroveň sa bude automaticky zvyšovať. Okrem toho nám počítačová hra dáva spätnú väzbu či sme uspeli, alebo nie a tým nás núti sa neustále zlepšovať a v prípade zlyhania, ktoré je oddeliteľnou súčasťou hry, opakovať hru a tým sa zlepšiť.
 \cite{cite14}
 %\footnote[9]{SALEN-TEKINBAS, K.: What is Game-Based Learning? . [online]. [2022-10-29]. Dostupné na: <https://www.q2l.org/about/>.}\\

 
 \indent Videohry môžu študentov vzdelávať v rôznych oblastiach, ako napríklad dejepis, geografia, matematika, jazyky a v mnohých ďalších. Študenti sa prostredníctvom hry napríklad prenesú do konkrétneho historického obdobia, kde si vyskúšajú určité role (poddaný, vodca...).\\
\indent V súčasnosti existuje niekoľko hier, ktoré podporujú vzdelávanie v dejepise, matematike, angličtine a pod. Jendou z nich je Assasin Creed. Assasin Creed  je historická hra, ktorej autor vydal viacero verzií. V súčasnosti sú najznámejšie tri, ktoré sa odohrávajú v Taliansku, Grécku a Egypte. Z definície historická hra, vychádza, že tvorca hry sa inšpiroval históriou miest, a tým sa študenti učia, ako to v daných historických mestách vyzerá, ako tam ľudia žijú a pod. Ak študent, ktorý Assasin Creed, podvedome vníma celé okolie, a tým pádom na hodine dejepisu sa mu vybavia, či už názvy miest, alebo znaky z daných období. \\
\indent Z daného jasne vyplýva, že úloha počítačových hier je určite dôležitá u študentov, či už z hľadiska vnútornej motivácie, intelektuálneho rozvoja, rozvoj kritického myslenia, pozornosti, riešenia problémov, ako aj rozvoj pamäte.\\
\indent Dokazuje to aj zaujímavý výskum, ktorý robili v Austrálii a zúčastnilo sa ho viac ako 12 000 študentov stredných škôl. Zistili, že študenti, ktorí periodicky hrali videohry, mali vyššie skóre z matematiky, prírodných vied a čítania, ako tí študenti, ktorí online videohry nehrali.
\cite{cite7}
%\footnote[10]{WIDMER, B.: Students Who Play Online Games Perform Better At School, Research Finds. [online]. [2022-10-30]. Dostupné na: < https://www.lifehack.org/451774/students-who-play-online-games-perform-better-at-school-research-finds>.} \\


\indent „Študenti, ktorí hrajú online hry takmer každý deň, dosahujú 15 bodov nad priemerom v matematike a 17 bodov nad priemerom vo vede, hovorí ekonóm Alberto Posso z RMIT University v Melbourne.“
\cite{cite7}
%\footnote[11]{WIDMER, B.: Students Who Play Online Games Perform Better At School, Research Finds. [online]. [2022-10-30]. Dostupné na: < https://www.lifehack.org/451774/students-who-play-online-games-perform-better-at-school-research-finds>.}\\

\indent Keď študent hrá hru, ktorá spĺňa kritéria uvedené v prvej kapitole, čo znamená, že v nej rieši veľký problém, pozostávajúci z malých úloh, aby sme sa posunuli na ďalšiu úroveň. To zahŕňa použitie niektorých všeobecných vedomostí a zručností z matematiky, prírodných vied, histórie, ktoré ste nadobudli. \\
\indent Pozoruhodné však je, že vedci si nie sú istí, prečo študenti, ktorí hrajú online videohry, dostávajú lepšie hodnotenie. Jeden z dôvodov môže byť to, že študenti, ktorí hrajú počítačové hry, lepšie prepájajú nadobudnuté poznatky a zároveň ich dennodenne využívajú, práve pri hraní online hier. \\ 	
\indent Posso a psycholog Peter Etchells sa vyjadrili, že danú oblasť je potrebné preskúmať a zrealizovať podrobnejší výskum. Dokonca niektoré pokusné výskumy potvrdzujú, že hranie počítačových hier podporuje schopnosť učenia sa, zlepšuje pamäť, motorické zručnosti a je možné ich použiť na zotavenie sa po poranení mozgu.
\cite{cite7}
%\footnote[12]{  WIDMER, B.: Students Who Play Online Games Perform Better At School, Research Finds. [online]. [2022-10-30]. Dostupné na: < https://www.lifehack.org/451774/students-who-play-online-games-perform-better-at-school-research-finds>.}  

Všetko je to vo vývoji a je potrebný podrobnejší výskum. \\
	\indent Podľa vlastnej skúsenosti viem, že hraním online hier, som sa výrazne zlepšil v anglickom jazyku. Hra bola postavená, tak, že som musel splniť určité misie. Daná hra bola v anglickom jazyku, ako väčšina dnešných známych hier, čiže som si musel viaceré názvy a frázy prekladať. Neskôr som sa ich naučil, pretože ak som chcel úspešne a rýchlo splniť misiu v hre, nechcel som sa zdržovať opätovným vyhľadávaním anglických výrazov. Následne som v škole na hodine angličtiny videl, že sa mi obohatila slovná zásoba dôsledkom hrania počítačových hier v anglickom jazyku. 








\section{Gamifikácia a jej úloha v študentskom učení }
\label{dolezitejsia} 
\noindent
\hspace*{1.5em}
Ako je už v prvej kapitole uvedené, pri gamifikácii „ide o využívanie herných mechaník v neherných oblastiach s cieľom odľahčiť určitú aktivitu a motivovať ľudí, aby sa do nej s nadšením pustili.“  Dôležité je si uvedomiť, že tak ako pri počítačových hrách aj pri gamifikácii musíme študentov správne namotivovať, a to prostredníctvom nástrojov gamifikácie.\\ 
\indent Medzi nástroje gamifikácie zaraďujeme mechanizmy a prvky hier. Procesu hry napomáhajú nasledujúce herné prvky:
\begin{itemize}
\item „pútavý príbeh,
\item vizualizácia – vzbudenie pozornosti
\item rozdelenie úloh na menšie úlohy
\item okamžitá spätná väzba
\item 	rôzne nezávislé poslania, ktoré nesúvisia s hlavným príbehom, ale ich splnenie navyšuje body hráča
\item 	body, odznaky, rebríček výsledkov – je veľmi dôležité, aby študenti mohli svoje výsledky navzájom porovnať
\item úrovne – ich splnenia poukazuj ich splnenia poukazujú na stupeň rozvoja, a spätná väzba napomáha k úspešnému zvládnutiu učiva. K úspechu metódy gamifikácie je nevyhnutné, aby boli prítomné mechanizmy hry.“
\cite{cite8}
%\footnote[14]{  MEZEIOVÁ, A.: Gamifikácia ako nástroj vzdelávania 21. storočia = Gamification as a tool for education of the 21st century. In: LÖSTER, T., LANGHAMROVÁ, J., VRABCOVÁ, J. (eds.): RELIK 2018 = reproduction of human capital - mutual links and connections : the 11th international scientific conference: the 11th international scientific conference, Praha : Oeconomica Publishing House, 2018, s. 274.}  

\end{itemize}
\noindent
\hspace*{1.5em} Práve vďaka vyššie spomenutým herným prvkom predídeme nude, ak by bola úroveň veľmi nízka alebo frustrácii, ak by úroveň bola nastavená veľmi vysoko. Akonáhle študenti prestanú úlohy v učení vnímať ako nutnosť (povinnosť), bude im to čo robia dávať zmysel, budú sa môcť posúvať vpred. To všetko prispeje k tomu, že študentov bude učenie baviť, a práve to je úlohou gamifikácie v študentskom učení.\\
\indent V súčasnosti gamifikáciu využívajú učitelia pri učení študentov, ale hlavne študenti pri učení sa, napríklad prostredníctvom rôznych aplikácií, či kvízov a súťaží na vysokej škole.\\
\indent Ako prvé uvediem aplikácie a hry, ktoré sú určené priamo na vzdelávanie, ale majú prvky gamifikácie v podobe získavania bodov, odznakov, posúvania sa na vyšší level a pod.
\begin{enumerate}
\item 	Math 180 – v rámci daného programu je množstvo možností, ako sa vzdelávať z matematiky. Poskytuje prispôsobený zoznam hier, vďaka ktorým je matematika zábavná a zároveň rozvíja výpočtovú a strategickú plynulosť študentov.
\cite{cite13}
%\footnote[15]{HMH MATH 180®. [online]. [2022-10-30]. Dostupné na: < https://www.hmhco.com/programs/math-180>.} 

\item 	MinecraftEdu – dokonca tvorcovia hry Minecraft, vytvorili MinecraftEdu, kde sa študenti vzdelávajú v oblasti architektúry. 
Druhou možnosťou ako využiť gamifikáciu, je pri učení študentov, či už na prednáškach alebo seminároch.
\item Vytvorenie avatarov – mnoho študentov hrá hry, ktoré umožňujú hráčovi vytvárať postavy (alter-ego), ktoré si môžu prispôsobiť a stavať na nich. Môžu ich upravovať podľa toho aké úlohy splnili. Každá ďalšia úroveň poskytuje väčšie množstvo výhod pre avatara. Taktiež môžeme rozvíjať rôzne zručnosti avatarov, z ktorých sa postupne stanú inžinieri, lekári a pod. \cite{cite1}
%\footnote[16]{  BLANKMAN, R.: The Fun of Learning: Gamification in Education. [online]. [2022-10-28]. Dostupné na: <https://www.hmhco.com/blog/what-is-gamification-in-education>.} 
\item	Odznaky a ocenenia – ide o rozdávania odznakov, prípadne ocenení, za splnenie úlohy. Úlohy by opäť mali byť stále viac náročné, ako pri online hrách, aby to študentov motivovalo a posúvalo ďalej. 
\item	Súťažné úlohy – súťaženie je výbornou formou, pretože pre študenta, ale aj celkovo človeka, je pocit víťazstva uspokojujúci. Tým pádom ich to núti premýšľať a zdolávať prekážky a vyriešiť problém. 
Tretie využitie gamifikácie je veľmi časté. Využívajú ho predovšetkým študenti vysokých škôl na internátoch. Uskutočňujú rôzne kvízy, či už z oblasti matematiky, histórie, geografie a pod., a tým pádom sú tu využité prvky gamifikácie ako získavanie bodov , rebríček výsledkov, plnenie úloh, spätná väzba a mnoho ďalších. 

\end{enumerate}






\section{Záver} 
\label{zaver} 
V práci som sa zameral predovšetkým na vysvetlenie úloh počítačových hier v študentskom učení a gamifikácie. Uviedol som ich prínos a využitie v živote študenta. Opísal som kritéria a prvky, ktoré by počítačové hry a gamifikácia mali spĺňať, aby mali zmysel a hodnotu v učení študentov. Zistil som, že online hry a gamifikácia ozvláštňujú učenie sa, je pútavejšie, zaujímavejšie, zábavnejšie a podporuje prepájanie nadobudnutých vedomostí. Keď študenti prestanú vnímať pri učení sa jednotlivé úlohy ako nudné a frustrujúce, budú sa posúvať vpred. To všetko prispeje k tomu, že študentov bude učenie baviť, a práve to je úlohou gamifikácie v študentskom učení. 
Počas písania článku, som taktiež zistil, že je potrebné vykonať podrobnejší výskum, o tom, ako dokážu počítačové hry pomôcť študentom v učení sa. Zároveň otvorenou otázkou ostáva, prečo študenti, ktorí hrajú vo vyššej miere počítačové hry, dosahujú lepšie výsledky v škole. 




%\acknowledgement{Ak niekomu chcete poďakovať\ldots}


% týmto sa generuje zoznam literatúry z obsahu súboru literatura.bib podľa toho, na čo sa v článku odkazujete
\newpage
\bibliography{literatura}
\bibliographystyle{plain} % prípadne alpha, abbrv alebo hociktorý iný
\end{document}
